\makeatletter
% ==========================================
% Define custom commands:

% This is a Zebra code listing (creates neat ``zebra''-like
% lines with numberings. You can define your own colors
% (modify the code_params.tex)
\newcommand\realnumberstyle[1]{#1}
\newcommand{\zebra}[3]{%
    {\realnumberstyle{#3}}%
    \begingroup
    \lst@basicstyle
    \ifodd\value{lstnumber}%
        \color{#1}%
    \else
        \color{#2}%
    \fi
        \rlap{\hspace*{\lst@numbersep}%
        \color@block{\linewidth}{\ht\strutbox}{\dp\strutbox}%
        }%
    \endgroup
}

% This one is for making spaces around the wrapfigures
% sometimes when the text is short, the wrapfigure will
% try to make everything after itself change its size.
% use \makespace to fix it
\newcommand{\makespace} {
  \par
  \loop
  \ifnum\c@WF@wrappedlines > 1
  \hspace*{1sp}\newline
  \advance\c@WF@wrappedlines by -1
  \repeat
  \par
}

% If you would like to get rid of some text, but still have it for future:
% Just put the text that you would like to hide into \ignore{your text}
\newcommand{\ignore}[1]{}

% This is a fixme request. In case of collaboration, this thing will emphasize
% the text and put a FIXME: keyword
\newcommand{\fixme}[1] {
  \textbf{\textcolor{red}{[FIXME: #1]}}
}

% This is a work in progress note. This will show a possible modification
% the text and WORKinPROGRESS: keyword:
\newcommand{\wip}[1] {
  \textit{\textcolor{blue}{[WORKinPROGRESS: #1]}}
}

% This is just a message about the end of section
\newcommand{\sectionend} {
  \begin{center}
    \textbf{END of SECTION}
    \pagebreak
  \end{center}
}

% End of commands definition
% ==========================================
\makeatother
