% ==========================================
% Define code styles:
\lstdefinestyle{custom_verilog}{
  % alsodigit={-},
  %alsoletter={`},
  basicstyle=\footnotesize\ttfamily,
  belowcaptionskip=0.1\baselineskip,
  breaklines=true,
  captionpos=t,
  commentstyle=\color{green!40!black},
  deletekeywords={...},         % Use it as a separator for keywords
  % Note that @ sign is better to use as an escape character 
  % but in Verilog it is already reserved:
  escapeinside={//^}{^},
  frame=L,
  identifierstyle=\color{black},
  keywordsprefix={`},
  %keywordsprefix=\#,
  keywordstyle=\bfseries\color{blue},
  language=Verilog,
  morecomment={
        [s][\color{red}]{/*!}{*/}
  },
  morekeywords={ % All keywords are in alphabetic order
        *,
        % Data type definitions:
        bit,
        electrical,
        logic,
        longint,
        shortint,
        shortreal,
        string,
        typedef,
        wreal,
        ...,
        % OOP constructs:
        class,
        endclass,
        endpackage,
        extends,
        import,
        package,
        super,
        this,
        virtual,
        ...,
        % blocks:
        analog,
        ...,
        % SystemVerilog specific:
        assert,
        endinterface,
        exclude,
        from,
        interface,
        set,
        ...
  },
  numbers=left,
  numberstyle=\zebra{gray!50}{white}, % Don't forget to declare ZEBRA command 
  otherkeywords={
        ::
  },
  showstringspaces=false,
  stepnumber=1,
  stringstyle=\color{orange},
  tabsize=2,
  % title=\lstname,
  xleftmargin=\parindent,
 }
% Change the code caption (Listing is kinda lame):
\renewcommand{\lstlistingname}{Code Listing}

\lstset{style=custom_verilog} % By default assume all codes are verilog type

% Make a blank line if some lines are skipped:
%\makeatletter
%\let\oldMSkipToFirst=\lst@MSkipToFirst
%\gdef\lst@MSkipToFirst{\oldMSkipToFirst}
%\makeatother
